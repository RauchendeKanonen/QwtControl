\documentclass[a4paper,12pt]{article}
\usepackage[latin1]{inputenc}
\usepackage{ngermanb}
\usepackage[ngermanb]{babel}
\usepackage{graphicx}

\usepackage[noindentafter]{titlesec}
\usepackage[latin1]{inputenc}
\usepackage{ngermanb}
\usepackage[ngermanb]{babel}
\usepackage{graphicx}
\usepackage{url}
\usepackage{hyperref}
\usepackage[noindentafter]{titlesec}
\usepackage{vhistory}
\usepackage{pdflscape} 
\usepackage{listings}
% paragraphs sehen aus wie subsubsubsections
\titleformat{\paragraph}[hang]{\bf}{\thetitle\quad}{0pt}{}						
\titlespacing{\paragraph}{0pt}{1em}{0.5em} 

% subparagraphs sehen aus wie vorher paragraphs
\titleformat{\subparagraph}[runin]{\bf}{}{0.5em}{}
\titlespacing{\subparagraph}{0pt}{1em}{1em}



%opening
\title{Time Domain Response}
\author{Florian Hillen}

\begin{document}
\section{Time-Domain Response}
The time domain response is generated by numerical inversion of the transfer function. 
Two algorithms have been implemented in laplace explorer yet. The Gaver-Stehfest and the week's method.
Gaver-Stehfest seems to be a robust and fast algorithm though is unable to process time-shifts accuratly.
Week's method is much more complicated and several parameters have to selected correctly to gain 
over Gaver-Stehfest.\\\\
Explanations to Gaver-Stehfest:\\
\url{http://www.cs.hs-rm.de/~weber/lapinv/gavsteh/gavsteh.htm}\\\\
Explanations to Week's algorithm:\\
\url{http://www.cs.hs-rm.de/~weber/lapinv/weeks/weeks.htm}\\\\
\\
Gaver-Stehfest is now not available any more.\\\\
After creating the time domain response the algorithm can be adjusted. By right-mouse click onto the Curve in
the list-view an pulldown-menu opens. By selecting setup, a setup-dialog opens.\\\\
\section{Week's algorithm}
$F(s) = \int e^{-st}f(t)dt$\\
$Re(s)>=\sigma_0$\\\\
$f(t)$ is the inverse Laplace transform of $F(s)$.  

\subsection{Abscissica ($\sigma_0$)}
The value $\sigma_0$ is referred to as the abscissa of 
convergence of the Laplace transform; it is the rightmost real part of the singularities of $F(s)$.\\\\
So for a stable system $\sigma_0$ can be selected $\sigma_0 >= 0$.
\subsection{Number of Laguerre expansion coefficients}
The number of Coeffizients is specified by $2^p$ where p has to be selected.
\subsection{Evaluation Pos}
$F(s + c)$ is evaluated where c has to be selected $c>0$ for singularities on/right of the imaginary axis.
\subsection{Scale Parameter}
For bettering the convergence.
\end{document}
