\documentclass[a4paper,12pt]{article}
\usepackage[latin1]{inputenc}
\usepackage{ngermanb}
\usepackage[ngermanb]{babel}
\usepackage{graphicx}

\usepackage[noindentafter]{titlesec}
\usepackage[latin1]{inputenc}
\usepackage{ngermanb}
\usepackage[ngermanb]{babel}
\usepackage{graphicx}

\usepackage[noindentafter]{titlesec}
\usepackage{vhistory}
\usepackage{pdflscape} 
\usepackage{listings}
% paragraphs sehen aus wie subsubsubsections
\titleformat{\paragraph}[hang]{\bf}{\thetitle\quad}{0pt}{}						
\titlespacing{\paragraph}{0pt}{1em}{0.5em} 

% subparagraphs sehen aus wie vorher paragraphs
\titleformat{\subparagraph}[runin]{\bf}{}{0.5em}{}
\titlespacing{\subparagraph}{0pt}{1em}{1em}



%opening
\title{Spezifikation, Validierung und Verifizierung der Maßnahme 2 für die Software des Medi Stream 3000}
\author{Florian Hillen}



\begin{document}
\section{Introduction}
Laplace-Explorer was made for exploring the properties of control systems. In the first development
stages, ist was only ment for visualizing the root locus of a close loop system. I found it
hard to to use a programm like scilab or matlab for doing this.
Later extension have been time-domain- and frequencyresponse plots.
All those views on one and the same system can be viewed at the same time. Parameter changes are actualized
to all views immediately.
The inversion of laplace back into the time domain is done numerically too. Two algorithms have been implemented
so far. Gaver-Stehfest and Week's method. See:
\begin{lstlisting}
http://www.cs.hs-rm.de/~weber/lapinv/weeks/weeks.htm
\end{lstlisting}

\subsection{root locus}
$F(s) = \frac{num(s)}{den(s)}$\\\\
The root locus of a closed loop system is extracted out of the carakteristic equation.\\\\
$den(s) = 0$.\\\\
Laplace-Explorer solves this by using Newton's (Madsen) methode.
So the solution is a numerical one.
\\
\subsubsection{Simple Example}
For example take the Plant as:\\\\
$G(s)=1/(s+1)^3$\\\\
And the controller as only proportional with gain k. So the open loop transfer function is:\\\\
$G_o(s)=k/(s+1)^3$\\\\
By closing the loop we get:\\\\
$G_c(s)=\frac{(k/(s+1)^3)}{(1+(k/(s+1)^3))}$\\\\
Add the Expression into Laplace Explorer:
\begin{lstlisting}
tf:(k/(s+1)^3)/(1+(k/(s+1)^3))
\end{lstlisting}
Where tf: names the expression as tf.
Then do a right-mouse click onto this expression and select in the pulldown-menu draw 
numeric root locus. A option dialog opens where you have to select a variable for 
parameter-sweep, the resolution and the range. By clicking Ok the Curve is added to the mainwindow
and a independent slider is added. By moving the slider from zero to some value, one can observe
the poles of the system moving toward the asymptotes. In the example above are 3 poles. 2 
(one conjugate complex) poles cross the imaginary axis when the gain $k>=8$. This is the point where the
System gets instable.

\subsubsection{Extendet Example}
For example take the Plant as:\\\\
$G(s)=1/(s+1)^3$\\\\
And the controller as proportional and integral with gain Kp and Ki:\\\\
$G_c(s)=Kp+Ki/s$\\\\
So the open loop transfer function is:\\\\
$G_o(s)=(Kp+Ki/s)/(s+1)^3$\\\\
By closing the loop we get:\\\\
$G_c(s)=\frac{((Kp+Ki/s)/(s+1)^3)}{(1+((Kp+Ki/s)/(s+1)^3))}$\\\\
Add the Expression into Laplace Explorer:
\begin{lstlisting}
tf:((Kp+Ki/s)/(s+1)^3)/(1+((Kp+Ki/s)/(s+1)^3))
\end{lstlisting}
In the root locus options dialog select Kp as Variable for the sweep and click ok.
Click onto the Variable $Ki$ and select add slider in the pulldown menu.
By varying Kp one can observe the roots moving on the path. By varying Ki, one can observe
the changing paths of the loci.

\subsection{Time domain response}
Take one of the examples above and modify the transfer function by multiplying it with $1/s$ (step response).\\\\
$S_r(s)=\frac{((Kp+Ki/s)/(s+1)^3)}{(1+((Kp+Ki/s)/(s+1)^3))}*1/s$\\\\
By right-mouse click onto this expression and selecting draw Response the response-options dialog opens.
Select the complex variable s as variable and click Ok. A new window with the response opens.

\subsection{Frequency response}
Take one of the examples above.\\\\
$S_r(s)=\frac{((Kp+Ki/s)/(s+1)^3)}{(1+((Kp+Ki/s)/(s+1)^3))}$\\\\
By right-mouse click onto this expression and selecting draw Bode the bode-options dialog opens.
Select the complex variable s as variable and click Ok. A new window with the response opens.

\end{document}